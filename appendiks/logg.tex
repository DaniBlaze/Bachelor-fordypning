
\section*{11. Januar: Bachelor seminar}
Hele gruppen møter til første seminar som gjelder praktisk info om oppgaveskriving. Vi er i gang. 

\section*{25. Februar: Etablerer kontakt med AS Rockwool}
Kom i kontakt med fabrikksjef Erik Ølstad i Rockwool’s avdeling i Moss. Kommer frem til en mulig problemstilling som gjelder virksomhetens nylige beslutning om å investere i ny elektrisk smelteovn. Sender mail til Espen denne kvelden for å få godkjent problemstillingen.

\section*{26. Februar: Mailkorrespondanse }
Får positivt signal fra Espen vedrørende problemstilling. Oppgaven er i gang.

\section*{1. Mars: Fremføring og veiledning}
Gruppen presenterer problemstillingen og mulige fremgangsmåter foran Espen og medelever. Ettersom vi er helt i startfasen benytter vi også muligheten til en veiledningstime. Espen belyser viktige momenter for oppgaveløsning.

\section*{24. April: Møte med Erik Ølstad:}
Møter fabrikksjef Erik Ølstad for å få svar og oppklaring i spørsmål som har dukket opp underveis. 

\section*{12. Mai: Besøker fabrikken i Moss}
Erik Ølstad guider oss gjennom fabrikken i Moss. Viser oss plantegning for den nye el-ovnen. Vi benytter også muligheten til å stille spørsmål om ting vi ønsker belyst. 

\section*{22. Mai: “Drop-in” veiledning} 
Etter å ha møtt på flere utfordringer møter ett av gruppemedlemmene på kontordøra til Espen. Espen tar seg tid til en kort veiledning og sentrale spørsmål rundt oppgaveløsningen blir belyst. 

\section*{3. Juni: Innlevering}
Tre år på Handelshøyskolen BI er historie i det vi leverer Bacheloroppgaven. Takk for oss!  
