For å vurdere lønnsomheten av investeringen vil vi sammenligne kontantstrømmene prosjektet vil generere med nåværende teknologiløsning. Kapittelet vil presentere faktorer som legges til grunn for netto nåverdiberegningene.

\section{Grunnlag for beregning av netto nåverdi med el-teknologi}
\subsection{ Investering og finansiering}
Investeringen i el-ovnen koster totalt 340 millioner kroner, hvor investeringsbeløpet fordeler seg over årene 2018, 2019 og 2020 med henholdsvis 10,203, 136,035 og 193,850 millioner kroner. Enova bidrar med 101,5 millioner kroner i finansiell støtte, som dekker ca 39,1\% av investeringsbeløpene i hvert av årene. Resterende beløp på 238,5 millioner kroner finansieres med interne midler fra Rockwool-konsernet. Selve ovnen har en pris på 130 millioner kroner, hvor resten er tilleggsinvesteringer knyttet til installasjon og tilpasning av ovnen. 

\indent \newline
I følge Rockwool har investeringen en teknisk levetid på 10 år og en økonomisk levetid på 20 år. Den økonomiske levetiden vil derfor legges til grunn i analysen. Investeringen vurderes til å ha en utrangeringsverdi på 20\%, det vil si 47,7 millioner kroner. I utgangspunktet ønsker ikke konsernet å selge teknologien til konkurrerende aktører, så en mulig alternativ anvendelse av ovnen vil være å flytte den til en annen fabrikk i konsernet. Vi velger allikevel å tillegge investeringen en utrangeringsverdi større enn null, da muligheten for salg foreligger. 

\indent \newline
Det avsettes tre måneder sommeren 2020 til installasjon og omstilling til ny produksjonsteknologi. For å analysere investeringen over 20 hele perioder gjør vi en forenkling og forutsetter at produksjon med el-teknologi er i gang fra og med 2021.

\subsection{Driftsinntekter}
Inntektsberegningene tar utgangspunkt i salgsinntektene fra 2018. Det forventes en årlig markedsvekst på 2\%. Det ligger dermed et potensiale i markedet for å øke markedsandelen. Vi legger til grunn en nedgang i salgsvolum på 0,5\% i 2019 og 2020. Årsaken til dette er at markedet etterspør grønnere produkter. Fra og med 2021 vil vi bruke en vekst i salgsvolum på 1,5\% de 10 første årene basert på at Rockwool vil levere produkter med  det laveste CO2-avtrykket i markedet. Etter dette forventes det at flere aktører vil ha gått over til mer miljøvennlig produksjonsteknologi og veksten settes derfor til 0,5\%.

\subsection{Driftskostnader}
Varekostnader består av råvarer, energi, bindemiddel, renhold, deponi, CO2-kvoter, transport og direkte produksjonslønn. For å synliggjøre forskjellene mellom den nye og nåværende smelteteknologien har vi valgt å trekke ut kostnadene knyttet til strøm, koks, CO2-kvoter og deponi. Pris per utslippskvote ligger per mai 2019 på 245 kroner, og i følge Rockwool vil denne kunne øke til 400 kroner innen 2030. Dette resulterer i en årlig vekst på 5,75\%. I 2018 hadde virksomheten et CO2-utslipp på 39.246 tonn med tilhørende antall tildelte kvoter på 25.249. Med en reduksjon i CO2-utslipp på rundt 80\% vil Rockwool ha et overskudd av tildelte klimakvoter. I utgangspunktet kunne de ha solgt disse til andre kvotepliktige virksomheter, men fra og med 2021 innføres en ny ordning hvor tildelte kvoter baseres på foregående års utslipp. Ekstra inntekter vil derfor ikke forekomme, kun bortfall av kostnader. Strømprisen tillegges en vekst på 0,5\% og i forhold til deponi legges det til grunn en reduksjon på 95\%. Det forutsettes at produksjonsvolumet endrer seg i takt med salgsveksten. Driftskostnadene følger produksjonsvolumet.

\indent \newline
Lønnskostnader tillegges en årlig vekst på 2\% basert på reallønnsprognoser fra SSB \cite{reallonnsprognoser}.
Andre driftskostnader består av vedlikehold og faste kostnader. Investeringen krever hyppigere vedlikehold og gir en økt kostnad på 509.090 kroner.

\section{Grunnlag for beregning av netto nåverdi med nåværende teknologi}
\subsection{Driftsinntekter}
Grunnet økt etterspørsel etter grønnere produkter legger vi til grunn en nedgang i salgsvolum på 0,5\% i 2019 og 2020. Deretter deles de neste 20 årene inn i 5-års intervaller, hvor salgsvolumet reduseres med henholdvis 2, 3, 5 og 10\% per år i hvert intervall. Beregningene er basert på samtaler med Rockwool og utviklingen i markedet. 

\subsection{Driftskostnader}
Ved å fortsette med nåværende produksjonsteknologi vil virksomheten ha økte kostnader knyttet til koks, CO2-kvoter og deponi. Priser og tilhørende vekst behandles under samme forutsetninger som i punkt 8.1.3. Koksprisen forventes å følge inflasjon.

\section{Avskrivninger}
De regnskapsmessige avskrivningene behandles som lineære avskrivninger over en levetid på 20 år. Utrangeringsverdien er satt til 20\% av investeringskostnaden. Skattemessig skal varige driftsmidler avskrives etter saldometoden jf. skatteloven §§14-40 og 14-41. Avskrivningssatsen er på 20\% jf. skatteloven §14-43 første ledd bokstav d. Videre følger det av skatteloven §14-42 andre ledd bokstav a at bidrag fra offentlig støtte skal trekkes fra kostprisen ved utregningen. Dette medfører dermed at utgangspunktet for saldoberegningen blir kostpris fratrukket tilleggsstøtte fra Enova. Saldoavskrivningene legges til grunn ved beregning av skatt \cite{skatteloven}. 

\indent \newline
Den nåværende smelteovnen er allerede ferdig avskrevet, og vi vil på bakgrunn av dette kun hensynta avskrivningene knyttet til el-ovnen, da andre avskrivninger vil være like ved begge alternativene.

\section{Skatt}
Selskapsskatten i Norge ble endret fra 23\% til 22\% for 2019. Det er usikkerhet knyttet til hvordan denne vil endres i fremtiden. Flere instanser, som for eksempel NHO, ønsker å redusere skatten ytterligere, og viser til at Norge har høyere selskapsskatt enn nabolandene våre. Til tross for at også selskapsskatten har hatt en nedadgående kurve de senere årene velger vi å legge til grunn 22\% gjennom hele investeringens levetid. 

\indent \newline
Skatten skal i utgangspunktet betales i to like terminer i løpet av første halvår etter inntektsåret. For kontantstrømmen totalt sett vil det ikke ha betydning om vi trekker skatten etterskuddsvis eller ikke. På bakgrunn av dette legger vi til grunn at skatten utbetales samme år som den oppstår \cite{skattaksje}.

\section{Inflasjon}
Vi har valgt å sette inflasjonen lik 2,0\% i henhold til Norges Bank sine prognoser for fremtidig inflasjonsøkning. Vi benytter oss av nominell metode og vil derfor inflasjonsjustere driftsinntekter og driftskostnader.

\section{Arbeidskapital}
Nødvendig arbeidskapital beregnes av neste års omsetning og settes til 5\%. Estimatet er utarbeidet på bakgrunn av gjennomsnittlig endring i arbeidskapital de siste 5 årene og samtaler med Rockwool.





 