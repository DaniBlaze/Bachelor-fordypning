På grunn av oppgavens omfang har vi måttet foreta en rekke beregninger som i virkeligheten er umulig å spå. I tillegg har vi inkludert de momentene vi fant hensiktsmessig for å besvare problemstillingen best mulig. Det betyr likevel ikke at vi har fanget opp alle momentene som har relevans for å svare best mulig på oppgavens formål. 

\indent \newline
Avkastningskravet er en av beregningene som vil gi størst utslag på netto nåverdien. I analysen beregnet vi beta til ROCKWOOL International. Vi brukte dermed det danske konsernets korrelasjon til den danske børsen, men med tilhørende norsk risikofri rente og markedspremie. Dette kan ha ført til unøyaktige estimater, noe vi prøvde å ta høyde for ved å gjøre en ekstra beregning for sammenlignbare selskaper. I avkastningskravet har vi også lagt på ytterligere risiko etter WACC- beregningen var utført. Disse ble gjort med delvis skjønn og delvis gjennom samtaler med Rockwool. Anslaget kan være unøyaktig, og vil i så fall gi prosjektet feil risiko. 

\indent \newline
Regnskapstallene fra proff.no skilte ikke mellom produksjonsfabrikken i Moss og Trondheim. Dette førte til at vi ble nødt til å ta en forutsetning om hvor store deler av inntektene og kostnadene som hørte til hver fabrikk. Samtaler med Erik Ølstad indikerte at forutsetningen vår om å tillegge Moss ca 70\% av produksjonen var et realistisk anslag. Dette dannet dermed utgangspunktet for analysen vår, men disse kan avvike fra virkeligheten.

\indent \newline
Estimering av inntekter og kostnader har vært blant de mest krevende anslagene knyttet til oppgaven. Gjennom analysen har vi sammenlignet dagens drift med den nye investeringen for å estimere merinntekter og merkostnader som følge av prosjektet. Hvordan disse prisene faktisk vil utvikle seg i fremtiden er umulig å anslå. Vi har prøvd å forholde oss konsistente ved begge modellene, men det kan likevel ikke utelukkes at vi i beregningene har vært “bias” mot ett av alternativene. Videre har vi tatt høyde for at begge alternativene har en relevant tidshorisont på 20 år. I realiteten er det nærliggende å anta at bedriften ikke ville drevet ulønnsomt over flere perioder uten å endre strategi eller legge ned. 

\indent \newline
Flere av beregningen våre har tatt utgangspunkt i historiske tall til tross for at historien ikke nødvendigvis former et realistisk bilde av fremtiden. Sensitivitetsanalysen har som formål å synliggjøre hvordan beregninger avviker fra forutsetningene vi har lagt til grunn for vår analyse. 

\indent \newline
Vi har i aller høyeste grad forsøkt å stille oss kritiske til eget arbeid gjennom hele arbeidsprosessen for å forsikre oss om at vi har kommet frem til hensiktsmessige beslutninger. 
