Kapittelet vil gi en kort presentasjon av virksomheten og markedet den opererer i. I tillegg vil det redegjøres for nåsituasjonen til Rockwool, og hvilke utfordringer virksomheten står overfor i dag.

\section{Om bedriften}
AS ROCKWOOL (senere i oppgaven omtalt som Rockwool) sin visjon er \textit{AS ROCKWOOL skal være ledende leverandør av isolasjon, der positivt bidrag til et bedre miljø og brannsikring skal være førende.} Virksomheten er et heleid norsk datterselskap av ROCKWOOL international A/S, hvor konsernet er verdens største steinullprodusent med ca 10.600 ansatte, salgskontorer i over 35 land, og 28 steinullfabrikker og 11 andre steinullrelaterte fabrikker i 17 land. Datterselskapet består av to fabrikker, lokalisert i henholdsvis Moss og Trondheim, og et salgskontor i Oslo. Med hjelp av 240 ansatte jobber virksomheten med utvinning av vulkansk stein for å produsere produkter, systemer og løsninger innenfor byggisolasjon. (https://www.rockwool.no/om-oss/)

\section{Historie}
I 1937 ble den første Rockwool-fabrikken etablert i Danmark. Få år senere utvidet konsernet med fabrikker i Larvik, trondheim og Moss. Siden oppstarten har Rockwool basert virksomheten på utvinning av vulkansk stein, hvor produksjonsprosessen har forandret seg lite. Imidlertid investerte de nærmere en halv milliard kroner i nytt produksjonsutstyr i 2002, med et formål om å automatisere produksjonen. Dette har ført til mer enn en fordobling av produksjonskapasiteten. I løpet av de siste årene har de innført en lean-metode som de kaller for Ropex, med et ønske om å effektivisere virksomheten gjennom hele verdikjeden.
(https://www.dagsavisen.no/moss/lokalt/viktig-for-industribyen-moss-1.316778)

\indent \newline
I 2016 vedtok ROCKWOOL-konsernet å forplikte seg til FNs bærekraftsmål, hvorav 6 av 10 er implementert som interne konsernmål for bærekraft. Målene representerer forbedringer innenfor sikkerhet og helse, vannforbruk, energieffektivitet, avfallsresirkulering, og reduksjon i avfall og CO2-utslipp i produksjonsprosessen. Ett av målene er å redusere CO2-utslippet med 10\% innen 2022 og 20\% innen 2030.

\section{Marked}  
Byggisolasjonsbransjen består av noen få store aktører som i likhet med Rockwool er datterselskaper av verdensomspennende konsern. Markedet karakteriseres av store mobilitetsbarrierer gjennom krav til kapitalintensive investeringer i spesialisert produksjonsutstyr. Rockwool har i flere år levert gode resultater, og er i dag markedets nest største aktør med en markedsandel på rundt 26\%. De mest nærliggende konkurrentene er Glava, Knauf, Sundolitt og Paroc, hvor Glava er markedets største med en markedsandel på 40\%. Kundene består i hovedsak av byggevarekjeder og entreprenører og sitter med høy forhandlingsmakt i form av at produsentene tilbyr lite differensierte produkter. Imidlertid leverer Rockwool og Paroc de mest differensierte produktene i form av produktegenskapene. Virksomhetene er de eneste isolasjonsprodusentene som leverer på alle punktene som gjelder (skriv om); produkter som isolerer, er vannavstøtende, har lyddempende egenskaper og er en god kilde til brannsikring. 

\indent \newline
Den siste tiden har Rockwool opplevd en lavere prosentvis vekst, grunnet en utvikling i markedet hvor miljøet blir vektlagt mer enn tidligere. Det blir vanligere for entreprenørene å BREEAM-sertifisere prosjektene sine. BREEAM er et miljøsertifiseringsverktøy for bygninger som legger vekt på miljøpåvirkning innenfor emner som energibruk, transport, materialer, avfall og forurensning. (https://byggalliansen.no/sertifisering/breeam/om-breeam-nor/) Dette påvirker spesielt produksjonsprosessen til isolasjonsprodusentene ved å stille krav til lavere utslipp. Rockwool sin nåværende smelteteknologi gir et høyere utslipp enn flere av konkurrentene, og er dermed en svakhet for virksomheten i forhold til å bli en foretrukken leverandør. 

\indent \newline
Det finnes også andre økonomiske insentiver for å redusere CO2-utslippet, da utslippskvoter står for en betydelig andel av produsentenes kostnader. Skrive litt mer, trenger mer info 
Markedsveksten har ligger på ca. 2\% de siste årene.

\indent \newline
Leverandører 
Kjøper koks fra Spania (må transporteres), vulkansk stein? Slagg? Annet?

\section{Elektrisk-ovn}
Kan være punktet bør flyttes til et annet sted og at alle punktene bør skrives om til ett punkt
\indent \newline
Rockwool besluttet i desember 2018 å investere i en ny smelteovn som vil benytte elektrisitet som energikilde. Beslutningen ble tatt etter å ha fått tilsagn om 101,5 millioner kroner i støtte fra Enova. Enova er forvalter av Energifondet, og støtter norske bedrifter som ønsker en omstilling til lavutslippssamfunnet. (https://www.enova.no/om-enova/) En elektrisk smelteovn er ikke tilgjengelig i markedet i dag, og investeringen krever at Rockwool selv utvikler nye og hensiktsmessige teknologiske løsninger tilpasset egen produksjon. Beregninger foretatt av selskapet viser til et investeringsbeløp på ca. 340 millioner kroner som fordeles i 2018, 2019 og 2020. Beløpet gjelder investeringer i innovasjon, teknologi og personalopplæring. 

\indent \newline
En el-ovn forventes å håndtere opp til 40\% gammel steinull (resirkulering), med en kapasitet på ca. 11.000 tonn steinullavfall fra byggeplasser. Dette tilsvarer mer enn totalt deponi av steinull per år fra bygg-markedet. Norge er i en særstilling når det gjelder avfall, hvor avfall til deponi, både fra produksjon og byggeplass, har vært relativt billig i flere år, men situasjonen er i ferd med å endre seg. Nye markeds- og myndighetskrav forventes i fremtiden. Avfallet fra produksjonen representerer stangmøllemel (granulert ull) og små mengder avfall/kapp av ny isolasjon som returneres fra markedet. Øvrig produksjonsavfall består av ovnsbunn (jern, slagger og fines) og flyveaske. Teknologien vil kunne føre til en avfallsreduksjon på 19.677 tonn per år. Det tilsvarer en reduksjon på ca. 95\% sammenlignet med 2017 (2018 tall?). 

\indent \newline
I tillegg unngås transport av stangmøllemel til Danmark og Movar som er en del av dagens løsning.  

\indent \newline
Konvertering fra koks til elektrisitet vil også føre til store endringer med tanke på CO2-utslippet. Analyser fra Rockwool viser til en potensiell reduksjon i CO2-nivå med ca. 80\%. Investeringen vil dermed kunne utgjøre store økonomiske utslag relatert til CO2-kvoter, innsparing i transport (transportkostnader relatert til kjøp av koks fra Spania bortfaller), avfall og resirkulering. 

\indent \newline
Sammenlikning av kupolovn med el-ovn: (bør kombinere alle tabellene i en)

\indent \newline
Investeringen baserer seg på utvikling innenfor teknologi og innovasjon i fire hovedelementer;

\begin{itemize}
\item[1.] \textbf{Sikkerhet} - Bygge den hittil største Submerged Arc Furnace (SAF) med lite smelteblad. En normal SAF-ovn med ønsket charge rate på 11,5 tonn per time vil ha en diameter på 8-9 meter og holde 189 tonn smelte. For å redusere risiko skal denne reduseres ned til 5,5-6 meter og holde 73 tonn smelte. Dette innebærer at en ny ovnstype med høyere “loadfaktor” må utvikles, noe som gir en høyere termisk belastning på ovnen og opp-muringsmaterialer.

\item[2.] \textbf{Resirkulering} - Utvikle en SAF som kan håndtere en høy last samtidig med høy resirkuleringsandel. Dagens el-ovner har en begrensning i load og resirkuleringsfaktor, det vil si enten en høy load og lav resirkulering eller lav load og høy resirkulering. En resirkuleringsandel på 40\% vil stille nye krav til røykgassrensning. Resirkulert steinull inneholder mer organisk materiale sammenlignet med kun bruk av stein som råvare. Mengden rørgasser forventes derfor å øke på grunn av dette sammenlignet med rørgasser fra dagens produksjon.

\item[3.] \textbf{Temperatur-stabilitet} - Utvikle en ny homogeniseringskanal. Fra ovn til spinnemaskiner er det over tre meter. Det ligger utfordringer i å sikre en stabil smeltetemperatur til spinnemaskinene. Det må derfor utvikles en homogeniseringskanal mellom ovn og eksisterende spinnere, da en slik kanal ikke er tilgjengelig i dag. Ved å sikre en stabil temperatur på +/- 10 grader celcius vil kanalen resultere i et høyere spinneutbytte.

\item[4.] \textbf{Liningsbyttet} - Utvikle rett sammensetning av ligning og effektivisere liningsbyttet. En høy grad av resirkulering vil stille nye krav til isolasjonsmateriale på innsiden av ovnen (lining). Et engineering team i Rockwool arbeider med å utvikle den rette sammensetningen for riktig ligning. Erfaring viser at resirkulering øker slitasjen på ligningen. Dette vil påvirke vedlikeholdsintervallene ved å gå fra hvert tredje år til hvert andre år. Målet er å redusere tiden det tar å skifte ligning, fra 3-4 uker til under 2 uker.
\end{itemize}
 
\subsection*{Risiko}
I forbindelse med dimensjoneringen av smelteovnen vil selskapet produsere i et kapasitetsområde som ikke er prøvd ut tidligere. Overdimensjoneringen vil øke investeringen og risikoen i prosjektet, men anses som nødvendig for å opprettholde produksjonskapasiteten i Moss. Hvis temperaturstabillitet ikke oppnås, kan avfallsprosenten stige eksplosivt til 10-15\%. En marginal endring på 1\% resulterer i økte kostnader på 1,2 millioner kroner. Dette er isolert sett den største risikoen i prosjektet.

\indent \newline
I tillegg påløper det risiko knyttet til drift i form av en situasjon hvor det ikke oppnås full effekt. Dette fører til at resten av produksjonsanlegget ikke anvendes optimalt i forhold til produksjonsvolumene det er tilpasset til. Det vil også med stor sannsynlighet oppstå hyppigere og lengre stopp i produksjon, spesielt med tanke på ligningslitasje. 

\indent \newline
En eventuell negativ effekt i forbindelse med investering i el-ovn og resirkulering vil være økt N2O. Dette er en kraftig drivhusgass som kan omregnes til CO2-ekvivalenter (was ist das? Bruker opp co2-kvoter?). 

\indent \newline
Oppsummert: risiko tilknyttet å nå full kapasitet, nede-tid og slitasje + ustabil smelte som fører til økt avfall

\subsection{Nåværende smelteteknologi}
Rockwool sin nåværende produksjonsprosess foregår ved at vulkansk stein smeltes ved høy temperatur i en smelteovn, og deretter spinnes til steinull. Steinullen blir videre omformet til ulike isolasjonsprodukter som skal isolere mot varme, kulde, brann og lyd. Smelteteknologien som blir brukt i dag er en kupolovn, hvor energibærerne består av koks, kalsinert karbon og SPL. Råvarer er Anortositt, Gabbro, Fundia slagg, Dolomitt og Merox slagg. Per i dag har ikke kupolovnen nødvendig teknologi til å håndtere avfall fra byggeplasser.
\indent \newline
Utslipp restfraksjoner for 2017 var som følger; må oppdateres til 2018 hvis mulig!
CO2: 39.246 tonn
Flyveaske: 1.034 tonn
Ovnsbunn (jern, slagger, fines) og stangmøllemel (granulert ull): 16.952 tonn
Restavfall ?

\indent \newline
Produksjonsvolum var på 50.909 tonn.
Omsetning 2017 - 818.052.000 kr
Resultat 2017 - 86.307.000 kr

\subsection*{Markedsrisiko}
Å basere produksjonen hovedsakelig på fossile energibærere kan være risikofullt og kostbart, da markedsutviklingen går mot grønnere produkter. Klimarisikoen kommer til syne både på et overordnet nivå, i form av Paris-avtalen, FNs bærekraftsmål og nasjonale målsettinger, og på et lavere nivå, i form av krav fra byggherrer og entreprenører. Rockwool har allerede erfart dette ved at byggaktørene har begynt å velge bort produkter som kan erstattes med produkter som har et lavere CO2-avtrykk. Dette er gitt at øvrige byggetekniske krav er ivaretatt og prisen er konkurransedyktig.  (skriv om).

\indent \newline
Fokus på sirkulær økonomi har også fått større betydning de senere årene. Byggavfall er en stor kilde til avfall, og det forventes at markedet på sikt vil stille strengere krav til materialgjenvinning. Statsbygg har eksempelvis signalisert at Rockwool må forberede seg på å endre dagens praksis og være i stand til å ta i mot retur av steinullavfall i fremtiden. 

\indent \newline
Avslutte med en kommentar om prognoser for vekst ved å fortsette som før.


\subsection{Brun smelteteknologi (BAT)}
En alternativ investering er en IMF-ovn (fluid bed ovn) hvor energibæreren er kull eller gass. Dette er en BAT-løsning (Best Available Technology) som kan bli gjennomført basert på Rockwool-konsernets egen smelteteknologi. Investeringen ligger på ca. 5 millioner kroner og innebærer redusert ovnsbunn og installasjon av full innvendig ligning. Kostnaden for en slik smeltelinje ligger på samme nivå som en elektrisk smeltelinje. Imidlertid vil en BAT-oppdatering maksimalt resultere i 20-30\% CO2-reduksjon. Dette er derfor ikke en bærekraftig investering med en 10-års horisont. Høye vedlikeholdskostnader vil i tillegg gjøre det vanskelig for Rockwool å finansiere en slik løsning, da kapasiteten i Moss ikke er stor nok.
