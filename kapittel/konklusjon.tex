Vi har utarbeidet en differansekontantstrøm over en periode på 20 år. Denne viser netto nåverdi av beslutningen om å investere i ny el-smelteovn, sammenlignet med å fortsette med nåværende smelteovn. Beregningene er basert på totalkapitalmetoden hvor relevant diskonteringsrente ble utarbeidet gjennom WACC med et risikopåslag knyttet til valuta og business risk. 

\indent \newline
Gjennom analysene våre har vi kommet frem til at investeringen vil være lønnsom, og generere en netto nåverdi på 588,500 millioner kroner. Den faktiske avkastningen på prosjektet er 28,32\% og gjenspeiler en avkastning på 19,57\% utover totalkapitalkravet. Sensitivitetsanalysen synliggjorde hvordan endringer i vekst og makroforhold vil påvirke prosjekts lønnsomhet. Resultatene viste til en svært solid investering, hvor det må forekomme ekstremutfall for at prosjektet ikke skal være lønnsomt. 

\indent \newline
Hovedårsaken til investeringens lønnsomhet ligger i markedsutviklingen, hvor prosjektet  legger til rette for at konsernets avdeling i Norge kan bevare dagens markedsposisjon og gi en stabil vekst i årene fremover. I alternativet med å fortsette med dagens løsning gir analysene grunn til å tro at Rockwool ville blitt en betydelig mindre aktør i det norske markedet.

\indent \newline
I tillegg til å generere en positiv netto nåverdi, vil pilotprosjektet i Moss styrke konsernets merkevare og støtte opp under selskapets utvalgte bærekraftsmål og visjon.

\indent \newline
Vi konkluderer med at beslutningen om å investere i en ny elektrisk smelteovn var lønnsom.
