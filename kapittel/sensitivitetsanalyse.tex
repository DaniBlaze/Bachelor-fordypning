Ved å ta en rekke forutsetninger om fremtidig utvikling, inntekter, kostnader og makroforhold har vi kommet frem til et resultat vi mener gjenspeiler nåverdien av investeringen. Likevel foreligger det stor usikkerhet rundt momentene fremvist i oppgaven. Det vil derfor presenteres ulike fremtidsbilder av investeringen med formål om å avdekke nåverdien gitt nye forutsetninger. Dette vil først belyses gjennom en \textit{best case} og \textit{worst case} analyse, før vi deretter undersøker hvor stor påvirkning de mest usikre momentene vil ha på nåverdien.

\subsection*{Best case}
I denne delen velger vi utelukkende å endre veksten i salgsvolumet. Grunnen til at vi velger å bruke salgsvolum som endringsvariabel er at den påvirker både inntektene og kostnadene, og beskriver den fremtidige utviklingen best. I best case analysen anslår vi at salgsvolumet øker med 3\% de første 10 årene, deretter vil salgsvolumet avta og ligge stabilt på 1\% gjenværende tid. Dette gir en økning i nåverdien på 340,491 millioner. Internrenten med denne forutsetningen økes fra 28\% til 36\%.

\subsection*{Worst case}
Worst case analysen legger også til grunn vekst i salgsvolum som endringsvariabel, men det forutsettes i tillegg en økning i strømprisen på 1\% per år. For å beskrive en fremtidig utvikling som vi anser som verst, men fortsatt realistisk, bruker vi en årlig negativ vekst i salgsvolumet på 2\% de første 10 årene, før den synker til 3\% per år resterende tid. Basert på disse estimatene vil investeringen gi en negativ netto nåverdi på 18,758 millioner kroner. 