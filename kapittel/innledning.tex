\section{Formål}
I denne oppgaven skal vi gjennomføre en investeringsanalyse av AS ROCKWOOL sin beslutning om å investere i en ny elektriske smelteovn. Formålet er å se hvorvidt dette er et lønnsomt prosjekt fra eiernes perspektiv ved tidspunktet for beslutningen. Vi vil gjøre leseren oppmerksom på at oppgaven vil analysere investeringen på bakgrunn av AS ROCKWOOL som et selvstendig selskap? (få frem at vi ikke legger vekt på at bedriften er en del av et konsern, evt finner vi ut av det senere, kanskje det går an å inkludere konsernet i oppgaven?)

\section{Problemstilling}
Problemstillingen vi ønsker å besvare er utarbeidet i samarbeid med AS ROCKWOOL, og lyder som følger:

\indent \newline
\textit{\textbf{Var investeringen av elektrisk smelteovn lønnsom på beslutningstidspunktet?}}

\indent \newline
Lønnsomhetsvurderingen vil baseres på en flerperiodisk nåverdianalyse med et formål om å maksimere eiernes interesser. I dette tilfellet vil investeringen ikke kun vurderes ut i fra et finansielt perspektiv, men også fra ROCKWOOL-konsernets eiere, som ønsker å implementere en mer miljøvennlig profil. Vurderingen vil gjennomføres på følgende grunnlag:

\begin{itemize}
\item Vurdere investeringen isolert sett.
\item Nåverdianalyse av virksomhetens fremtidige kontantstrømmer ved beslutning om å investere i ny produksjonsteknologi.
\item Nåverdianalyse av virksomhetens fremtidige kontantstrømmer ved beslutning om å ikke investere i ny. produksjonsteknologi (fortsette som før).
\item Nåverdianalyse av virksomhetens fremtidige kontantstrømmer ved beslutning om å investere i alternativt “grønt” produksjonsutstyr (BAT-løsning/brun energi).
\end{itemize}

\indent \newline
I tillegg vil vi gjennom scenario(scenarie?)-analyse skissere tre forskjellige fremtider for investeringen (best, basis og worst), for å gjøre ledelsen mest mulig forberedt på fremtidige beslutninger relatert til virksomhetens strategi. (kanskje legge til noe om støtte fra Enova også?) kan være at dette punktet skal smelles inn et annet sted også!