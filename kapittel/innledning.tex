\section{Formål}
I denne oppgaven skal vi gjennomføre en investeringsanalyse av ROCKWOOL International sin beslutning om å investere i en ny elektriske smelteovn på fabrikken i Moss. Formålet er å se hvorvidt dette er et lønnsomt prosjekt fra eiernes perspektiv ved å analysere merverdien av investeringen sett opp mot nåværende produksjonsteknologi. For å vurdere lønnsomheten vil vi beregne en differansekontantstrøm basert på totalkapitalmetoden som neddiskonteres med relevant avkastningskrav. Analysen utføres for Rockwool-konsernet, men vil hovedsakelig fokusere på datterselskapet i Norge og fabrikken i Moss.

\section{Problemstilling}
Problemstillingen vi ønsker å besvare er utarbeidet i samarbeid med Rockwool, og lyder som følger:

\indent \newline
\textit{\textbf{Var beslutningen om å investere i en elektrisk smelteovn lønnsom?}}

\indent \newline
Lønnsomhetsvurderingen baseres på en flerperiodisk netto nåverdianalyse med formål om å maksimere eiernes interesser. Vurderingen vil gjennomføres på følgende grunnlag:

\begin{itemize}
\item Beregne fremtidige kontantstrømmer ved beslutning om å fortsette med nåværende produksjonsteknologi.
\item Beregne fremtidige kontantstrømmer ved beslutning om å investere i ny produksjonsteknologi.
\item Beregne differansekontantstrømmer som neddiskonteres med relevant totalkapitalkrav.
\end{itemize}

\indent \newline
I tillegg vil vi gjennom en sensitivitetsanalyse belyse hvordan endringer i ulike variabler og forutsetninger vil påvirke netto nåverdien. Analysen vil også ta for seg et \textit{best-} og \textit{worst} case scenario for å gjøre selskapets ledelse bevisst på utfallsrommet investeringen befinner seg i.