Metode omhandler aspekter knyttet til hvordan man går frem for å tilegne seg kunnskap \cite{genaro}. Metode er viktig i utredningen av økonomiske problemstillinger/analyser ettersom det spiller en sentral rolle i forberedelsene, gjennomføringen og tolkningen av undersøkelsene. I tillegg til å sikre god gjennomføring av egne analyser skal metodelæren bidra til å kunne evaluere styrker og svakheter ved andres undersøkelser. 

\indent \newline
Metodelæren skilles i kvantitativ og kvalitativ metode. Kvantitativ metode tar sikte på å forklare eller anslå, mens kvalitativ metode tar sikte på å forstå.Vi har benyttet oss av primær- og sekundærdata som er innhentet gjennom kvantitative og kvalitative metoder. Primærdata er data som blir innhentet til et spesifikt formål, mens sekundærdata er data som allerede eksisterer og gjerne har tjent et annet formål.
 
\section{Kvantitativ metode}
For å utarbeide en netto nåverdi-analyse har vi vært avhengig av historiske regnskapstall og tekniske data knyttet til produksjon og anslag rundt den nye smelteovnen. Rockwool har bistått med informasjon relatert til teknisk data. Vi har benyttet dataene for å modellere kontantstrømmer, avkastningskrav og sensitivitetsanalyser i Excel. Foruten innsikten fra Rockwool har vi brukt Proff-Forvalt, Norges Bank, NVE, samt flere relevante nettsider for å redegjøre for makroforhold og andre relevante beregninger.
 
\section{Kvalitativ metode}
Erik Ølstad, fabrikksjef i Moss, har vært vår kilde for innhenting av primærdata. Vi har ved flere anledninger møttes for uformelle samtaler der vi har fått belyst relevante spørsmål. Innsikten fra Erik har vært svært hjelpsom, både i form av kvantitative analyser, men han har også gitt oss en solid forståelse for markedet Rockwool opererer i.
