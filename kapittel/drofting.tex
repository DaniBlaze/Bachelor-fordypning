I denne investeringsanalysen har vi beregnet netto nåverdi for to ulike alternativer for ROCKWOOL International, med et formål om å avgjøre om investeringen i ny el-smelteovn er lønnsom. Beregningene viser til en ekstraordinær avkastning på 19,57\% utover totalkapitalkravet, sammenlignet med alternativet som ville vært å fortsette med nåværende produksjonsteknologi. I et effisient marked er det i utgangspunktet ikke mulig å oppnå en slik ekstraordinær avkastning, og vi vil i det følgende diskutere årsaker som ligger bak dette, samt faktorer som er essensielle for lønnsomheten til prosjektet. 

\indent \newline
En viktig forutsetning for investeringens lønnsomhet er utviklingen i markedet. Det økende klima-fokuset, i form av FNs bærekraftsmål og Parisavtalen, skaper press på aktørene til å redusere utslipp i produksjonsprosessen. Dette forsterkes i tillegg av byggentreprenørenes økende bruk av BREEAM-sertifisering. Investeringen kan på grunnlag av dette karakteriseres som en strategisk beslutning med formål om å beholde dagens markedsandel og posisjon i Norge. Markedet er inne i en utvikling hvor produsenter som ikke tilpasser virksomheten til endringene vil kunne oppleve dramatiske fall i salgsveksten. Dette omtales ofte som “stall points”, og oppstår når aktører med gode markedsposisjoner ikke evner å oppfatte endring i markedsforutsetningene. Dette gjelder spesielt endring i kundenes verdivurdering av produktegenskapene. 

\indent \newline
Grunnen til at fabrikken i Moss ble valgt ut som pilotprosjekt er at det norske markedet oppfattes som det med størst press på de nevnte faktorene. I tillegg legger den norske politikken til rette for at virksomheter som ønsker å redusere utslippene kan få finansiell hjelp og støtte til å gjennomføre dette. I denne sammenheng var støtten fra Enova en viktig del av beslutningen. Sensitivitetsanalysen viser imidlertid at prosjektet gir en positiv netto nåverdi på 499,184 millioner kroner uten støtten på 101,5 millioner kroner, og er derfor ikke avgjørende for at prosjektet skal vurderes som lønnsomt. 

\indent \newline
Hovedårsaken til de store forskjellene i fremtidig kontantstrøm de to alternativene genererer er først og fremst veksten som legges til grunn. I alternativet med å fortsette med nåværende produksjonsteknologi vil virksomheten i Norge reduseres til en betydelig mindre aktør i markedet, mens den nye el-teknologien vil sørge for en fortsettelse av den stabile veksten virksomheten har opplevd i en årrekke. Det er også viktig å se på hvilke konkurransefortrinn investeringen gir. Sammenlignet med de andre konkurrentene vil Rockwool minimum redusere CO2-utslippet til samme nivå som Glava (markedsleder) ligger på i dag. De vil også oppnå et konkurransefortrinn i forhold til reduserte kostnader relatert til deponi. I dag prøver virksomheten å levere det mest differensierte produktet i form av produktegenskapene. Kostnadsbesparelsene vil kunne åpne opp for en ny virksomhetsstrategi i form av kostnadslederskap. Virksomheten vil derfor ha gode forutsetninger for å utfordre konkurrentene på pris.

\indent \newline
Imidlertid er det flere usikkerhetsmomenter knyttet til den nye teknologien, som kan påvirke lønnsomheten negativt. Innkjøringsperioden forventes å ta tre måneder, men siden teknologien ikke er testet ut i like stor skala tidligere, ligger det et usikkerhetsmoment ved at tilpasningen kan ta lenger tid enn forventet. Dette gjelder spesielt med tanke på å oppnå effektiv utnyttelse av ovnen og samsvar med øvrige deler av produksjonsanlegget. For å synliggjøre usikkerheten i prosjektet har vi gjennom sensitivitetsanalysen belyst hvordan endringer i de subjektive forutsetningene vil påvirke investeringens lønnsomhet. Det er vanskelig å forutse fremtidig utvikling i makrofaktorene, men analysen viser at det må forekomme ekstremutfall i forutsetningene for å gi en negativ netto nåverdi. Vi anser derfor investeringen som svært robust for uforutsette utfall.

\indent \newline
Avslutningsvis vil vi påpeke at investeringen kan skape verdier utover den positive netto nåverdien. Hvis pilotprosjektet viser seg å bli suksessfullt, vil investeringen gi selskapet verdifull kunnskap og erfaring til å imøtekomme de globale grønne markedsendringene. Investeringen kan potensielt være begynnelsen på endring av produksjonsteknologi blant alle fabrikkene til konsernet. Med en lenger tidshorisont enn vi har lagt til grunn i oppgaven forventes det et økende miljøpress i flere land. Vi anser derfor investeringen som en viktig beslutning med tanke på selskapets fremtidige utvikling, vekst og merkevare.
