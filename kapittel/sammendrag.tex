I desember 2018 besluttet ROCKWOOL International å investere i en ny elektrisk smelteovn på fabrikken i Moss. Konsernet produserer isolasjonsprodukter som utvinnes av vulkansk stein, og er verdens ledende leverandør av produkter og løsninger basert på steinull. Formålet med oppgaven er å utføre en investeringsanalyse på vegne av selskapet for å undersøke om prosjektet er lønnsomt. 

\indent \newline
For å vurdere lønnsomheten av prosjektet benytter vi oss av netto nåverdimetoden. Beregningen er basert på en differansekontantstrøm som er utarbeidet ved å sammenligne nåværende smelteteknologi med den nye elektriske smelteovnen. Kontantstrømmene neddiskonteres med relevant avkastningskrav justert for valutarisiko og business risk. Avkastningskravet er estimert gjennom beregning av selskapets egenkapitalbeta som brukes til å beregne egenkapitalkrav og selskapets totalkapitalkrav.

\indent \newline
I oppgaven drøfter vi ulike makroforhold som legges til grunn for fremtidig utvikling i isolasjonsbransjen. Den viktigste faktoren er det globale fokuset mot en grønnere fremtid. Analysen viser til flere usikkerhetsmomenter som er vanskelig å forutse hvordan vil utvikle seg i fremtiden. Sensitivitetsanalyser er derfor benyttet for å belyse hvordan endringer i forutsetningene vil påvirke netto nåverdi. I tillegg har vi utført en \textit{best-} og \textit{worst} case analyse for å gjøre ledelsen i selskapet bevisst på utfallsrommet investeringen befinner seg i.

\indent \newline
Differansekontantstrømmen  viser til en positiv netto nåverdi på 588,500 millioner kroner og gir støtte til å konkludere med at investeringen er lønnsom. I tillegg vil prosjektet styrke selskapets merkevare og gi bedre forutsetninger til å imøtekomme endringen i markedsutviklingen.
